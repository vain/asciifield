\documentclass[a4paper,parskip=full*]{scrartcl}
\usepackage[utf8]{inputenc}
\usepackage[T1]{fontenc}
\usepackage[ngerman]{babel}
\usepackage{lmodern}

\usepackage{amsmath}            % Umgebungen wie align, ...
\usepackage{amsfonts}           % beinhaltet z.B. Fraktur-Schrift
\usepackage{amssymb}            % weitere Symbole

\setlength{\jot}{12pt}          % Größerer Zeilenabstand in align

\usepackage{tikz}

\newcommand{\bvec}[1]{\ensuremath{\mathbf{#1}}}
\newcommand\numberthis{\addtocounter{equation}{1}\tag{\theequation}}

\begin{document}

\begin{Huge}
	\begin{center}
		Herleitung der perspektivischen Projektionsmatrix
	\end{center}
\end{Huge}

\begin{center}
	-- Peter Hofmann, Februar 2015 --
\end{center}

\tableofcontents

\section{Koordinatensystem, Blickrichtung, Objektpositionen}
Wir verwenden ein rechtshändiges Koordinatensystem wie in
Abbildung~\ref{coord}. Die Kamera wird im Ursprung platziert und
blickt in \emph{negative} $z$-Richtung, damit die positive $x$-Richtung
weiterhin nach "`rechts"' und positive $y$-Richtung nach "`oben"' zeigt.
Objekte werden daher ebenfalls mit \emph{negativen} $z$-Koordinaten
platziert.

\begin{figure}[h]
	\centering
	\begin{tikzpicture}
	\draw[<->]
		(0, 2) node[above] {$y$}
		-- (0, 0)
		-- (2, 0) node[right] {$x$};
	\draw[->]
		(0, 0)
		-- (-1, -1) node[left] {$z$};
\end{tikzpicture}

	\caption%
	{Rechtshändiges Koordinatensystem.}

	\label{coord}
\end{figure}

\section{Herleitung der Projektionsmatrix mit Kegelstumpf}
Wir wollen eine Zentralprojektion erreichen mit einem Fluchtpunkt.
Ähnlich der Konventionen bei OpenGL verwenden wir dafür einen
Kegelstumpf, den wir in einen Einheitswürfel transformieren.

\subsection{Grundlegende Projektionsmatrix: Zentralprojektion}
In $z$-Richtung wird dieser Kegelstumpf durch die beiden Werte $n$ und
$f$ für "`near"' und "`far"' begrenzt. Beide Werte werden negativ sein
und es gilt $n > f$ ($n$ ist also näher an der Null). Bei $n$ wird
unsere Bildebene liegen. Alle Objekte werden in $z$-Richtung zwischen
$n$ und $f$ erwartet. Ein Beispiel für diesen Aufbau ist in
Abbildung~\ref{frustumz} zu sehen.

\begin{figure}[h]
	\centering
	\begin{tikzpicture}
	\draw[shift={(-1.5, -1.5)},<->]
		(0, 0.5) node[above] {$y$}
		-- (0, 0)
		-- (-0.5, 0) node[left] {$z$};

	\begin{scope}[scale=0.5]
		\draw (0, 0) -- (7.5, 0);
		\draw[thick] (0, 0) -- (7.5, 5);
		\draw[thick] (0, 0) -- (7.5, -5);

		\draw (3, 2) -- (3, -2);
		\draw (6, 4) -- (6, -4);

		\draw[fill=red] (0, 0) node[left] {$(0, 0)$} circle (2pt);
		\draw[fill=red] (3, 0) node[below left] {$n$} circle (2pt);
		\draw[fill=red] (6, 0) node[below left] {$f$} circle (2pt);
	\end{scope}
\end{tikzpicture}

	\caption%
	{Querschnitt des Kegelstumpfs mit $n$ und $f$.}

	\label{frustumz}
\end{figure}

\begin{figure}[h]
	\centering
	\begin{tikzpicture}
	\draw[shift={(-1.5, -1.5)},<->]
		(0, 0.5) node[above] {$y$}
		-- (0, 0)
		-- (-0.5, 0) node[left] {$z$};

	\begin{scope}
		\draw (0, 0) -- (7.5, 0);
		\draw[thick] (0, 0) -- (7.5, 5);
		\draw[thick] (0, 0) -- (7.5, -5);

		\draw (3, 2) -- (3, -2);
		\draw (6, 4) -- (6, -4);
		\draw[dotted] (0, 0) -- (4.5, 1.5);
		\draw[dotted] (4.5, 0) -- (4.5, 1.5);

		\draw[fill=red] (0, 0) node[left] {$(0, 0)$} circle (2pt);
		\draw[fill=red] (3, 0) node[below left] {$n$} circle (2pt);
		\draw[fill=red] (6, 0) node[below left] {$f$} circle (2pt);

		\draw[fill=red] (4.5, 1.5) circle (2pt);
		\draw[fill=red] (3, 1) circle (2pt);
		\draw[fill=red] (4.5, 1.5) node[above right] {$\bvec{p}$};
		\draw[fill=red] (4.5, 0) node[below left] {$\bvec{p}_z$} circle (2pt);
		\draw[fill=red] (3, 1) node[above right] {$\bvec{p}'$};
	\end{scope}
\end{tikzpicture}

	\caption%
	{Gegebener Punkt $\bvec{p}$ und gesuchter Punkt $\bvec{p}'$.}

	\label{projy}
\end{figure}

Wird nun ein Punkt $\bvec{p}$ projiziert, dann suchen wir den
dazugehörigen Punkt $\bvec{p}'$, der sich durch die Zentralprojektion
ergibt. In Abbildung~\ref{projy} kann man erkennen, dass sich hier der
zweite Strahlensatz anwenden lässt:
\begin{align*}
	\frac{\bvec{p}_y}{\bvec{p}_z} &= \frac{\bvec{p}'_y}{n}  \\
              \iff ~~ \bvec{p}'_y &= \bvec{p}_y \cdot \frac{n}{\bvec{p}_z}
\end{align*}
Für die nicht eingezeichnete $x$-Richtung gilt dasselbe:
\begin{align*}
	\frac{\bvec{p}_x}{\bvec{p}_z} &= \frac{\bvec{p}'_x}{n}  \\
              \iff ~~ \bvec{p}'_x &= \bvec{p}_x \cdot \frac{n}{\bvec{p}_z}
\end{align*}
Daraus ergibt sich unter Verwendung von homogenen Koordinaten
vorübergehend die Matrix $M_1$:
\begin{align*}
	M_1 &= 
	\begin{pmatrix}
		1 & 0 & 0 & 0  \\
		0 & 1 & 0 & 0  \\
		0 & 0 & 1 & 0  \\
		0 & 0 & \frac{1}{n} & 0
	\end{pmatrix}
\end{align*}
Beispielhafte Multiplikation:
\begin{align*}
	M_1 \cdot \begin{pmatrix} x \\ y \\ z \\ 1 \end{pmatrix} =
	\begin{pmatrix} x \\ y \\ z \\ \frac{z}{n} \end{pmatrix} \cong
	\begin{pmatrix} x \frac{n}{z} \\ y \frac{n}{z} \\ n \end{pmatrix}
\end{align*}
Wie gewünscht wurden $x$ und $y$ also jeweils mit dem Faktor
$\frac{n}{z}$ multipliziert.

\subsection{Transformation in den Einheitswürfel}
\subsubsection{Definition des Bildbereichs}
Bisher wurde lediglich eine Zentralprojektion erreicht. Das heißt, dass
jeder Punkt im Raum auf unsere Bildebene projiziert werden kann. Eine
Ebene ist unendlich groß, unser Bildschirm aber endlich. Daher wollen
wir nun auf der Bildebene ein Rechteck für den tatsächlichen Bildbereich
definieren. Erst dadurch entsteht auch tatsächlich ein Kegelstumpf.

In $x$-Richtung wird der Bildbereich durch $l$ und $r$ für "`left"' und
"`right"' beschränkt, in $y$-Richtung durch $b$ und $t$ für "`bottom"'
und "`top"'.

In den allermeisten Fällen wird dieses Rechteck im Ursprung zentriert
sein. Wir betrachten jedoch zunächst den allgemeinen Fall. Für den
genannten Sonderfall bietet sich ohnehin die Vereinfachung an, die im
nächsten Abschnitt beschrieben wird.

\begin{figure}[h]
	\centering
	\begin{tikzpicture}
	\draw[shift={(-2.5, -2.5)},<->]
		(0, 0.5) node[above] {$y$}
		-- (0, 0)
		-- (0.5, 0) node[right] {$x$};

	\draw[fill=red] (-1, -1) node[left] {$(0, 0)$} circle (2pt);

	\begin{scope}[shift={(-1, -1)}]
		\draw (0, 0) -- (6, 0);
		\draw (0, 0) -- (0, 4);

		\draw[dashed] (1, 0) -- (1, 4);
		\draw[dashed] (5, 0) -- (5, 4);

		\draw[dashed] (0, 1) -- (6, 1);
		\draw[dashed] (0, 3) -- (6, 3);

		\draw (1, 0) node[below] {$l$};
		\draw (5, 0) node[below] {$r$};
		\draw (0, 1) node[left] {$b$};
		\draw (0, 3) node[left] {$t$};
	\end{scope}

	\draw (0, 0) rectangle (4, 2);
\end{tikzpicture}

	\caption%
	{Bildbereich auf der Projektionsebene.}

	\label{lrtb}
\end{figure}

\subsubsection{Bestimmung der Einträge für $x$ und $y$}
Sei zur Steigerung der Lesbarkeit nun $y$ die $y$-Koordinate des
Ausgangspunkts $\bvec{p}$ und $y'$ die $y$-Koordinate des auf die
Bildebene projizierten Punkts $\bvec{p}'$. Wir möchten diejenigen Punkte
in das Intervall $[-1, 1]$ überführen, die innerhalb des rechteckigen
Bildbereichs liegen, also suchen wir das skalierte $y''$:
\begin{align*}
	y' \in [b, t]  ~~  \rightarrow  ~~  y'' \in [-1, 1]
\end{align*}
Wir wissen, dass es eine lineare Abbildung ist und dass $b$ auf $-1$
abgebildet wird und $t$ auf $1$. Daher können wir das folgende LGS
aufstellen:
\begin{align*}
	-1 &= \alpha \cdot b + \beta ~ , \numberthis\label{lgs1} \\
	 1 &= \alpha \cdot t + \beta  \numberthis\label{lgs2}
\end{align*}
Aus \eqref{lgs2} folgt direkt:
\begin{align*}
	\beta &= 1 - \alpha \cdot t  \numberthis\label{lgs3}
\end{align*}
Setze \eqref{lgs3} in \eqref{lgs1} ein:
\begin{align*}
	-1 &= \alpha \cdot b + 1 - \alpha \cdot t  \\
	   &= \alpha \cdot (b - t) + 1  \\
	\Rightarrow ~~ \alpha &= \frac{-2}{b - t}  \numberthis\label{lgs4}
\end{align*}
Dann \eqref{lgs4} in \eqref{lgs3}:
\begin{align*}
	\beta &= 1 - \frac{-2}{b - t} \cdot t  \\
	      &= \frac{b - t}{b - t} - \frac{-2 \cdot t}{b - t}  \\
	      &= \frac{b - t + 2 \cdot t}{b - t}  \\
	      &= \frac{b + t}{b - t}
\end{align*}
Das LGS ist gelöst und insgesamt ergibt sich:
\begin{align*}
	y'' = \frac{-2}{b - t} \cdot y' + \frac{b + t}{b - t}
\end{align*}
Der letzte Schritt ist, $y' = y \cdot \frac{n}{z}$ einzusetzen und die
Gleichung dann in eine für die Matrix geeignete Form zu bringen. Das
heißt, es muss eine Summe aus $x$, $y$ und $z$ sein, wobei diese Werte
noch einfache Koeffizienten haben dürfen, und die Summe kann am Ende
eventuell noch durch einen Wert geteilt werden (aufgrund der homogenen
Koordinaten).\footnote{Es gibt hier häufig die Konvention, dass durch
	$-z$ geteilt werden soll. Das machen wir an dieser Stelle
\emph{nicht}.} Das ist Schritt für Schritt:
\begin{align*}
	y'' &= \frac{-2}{b - t} \cdot \left( y \cdot \frac{n}{z} \right) + \frac{b + t}{b - t}  \\
	    &= \frac{-2}{b - t} \cdot y \cdot \frac{n}{z} + \frac{b + t}{b - t} \cdot \frac{z}{z}  \\
	    &= \left( \frac{-2 \cdot n}{b - t} \cdot y \right) \cdot \frac{1}{z} + \left( \frac{b + t}{b - t} \cdot z \right) \cdot \frac{1}{z}  \\
	    &= \left( \frac{-2 \cdot n}{b - t} \cdot y + \frac{b + t}{b - t} \cdot z \right) \cdot \frac{1}{z}  \numberthis\label{yscaled}
\end{align*}
Gleichung~\eqref{yscaled} genügt unseren Anforderungen.

Dieser Prozess lässt sich analog auch für $x'$ beziehungsweise $x''$
durchführen. Hier ist die Forderung:
\begin{align*}
	x' \in [l, r]  ~~  \rightarrow  ~~  x'' \in [-1, 1]
\end{align*}
Das Ergebnis ist genau dasselbe, nur $l$ statt $b$ und $r$ statt $t$.

Damit ist unsere Matrix vorläufig diese:
\begin{align*}
	M_2 =
	\begin{pmatrix}
		\frac{-2n}{l - r} & 0 & \frac{l + r}{l - r} & 0  \\
		0 & \frac{-2n}{b - t} & \frac{b + t}{b - t} & 0  \\
		? & ? & ? & ?  \\
		0 & 0 & 1 & 0
	\end{pmatrix}
\end{align*}

\subsubsection{Bestimmung der Einträge für $z$}
Die dritte Zeile ist bisher unbestimmt, weil wir uns noch nicht darum
gekümmert haben, wie $z$ abgebildet werden soll. Das werden wir nun
nachholen.

$z$ wird nicht projiziert, also ist $z' = z$. Das einzige, was hier
passieren soll, ist eine Transformation von $z' \in [n, f]$ nach $z''
\in [-1, 1]$, damit wir in der Gesamtheit einen Einheitswürfel
erreichen. Wir wissen, dass dafür weder $x$ noch $y$ relevant sind,
also sind $m_{3,1} = m_{3,2} = 0$. Die beiden anderen Einträge sind noch
offen.

Betrachten wir noch einmal die Multiplikation mit einem homogenen
Vektor und legen den Fokus auf das Endergebnis der $z$-Komponente nach
Dehomogenisierung:
\begin{align*}
	M_2 \cdot \bvec{p} &=
	\begin{pmatrix}
		\dots & \dots & \dots & \dots  \\
		\dots & \dots & \dots & \dots  \\
		0 & 0 & \alpha & \beta  \\
		0 & 0 & 1 & 0
	\end{pmatrix}
	\cdot
	\begin{pmatrix}
		\dots \\ \dots \\ z \\ 1
	\end{pmatrix}, \\
	\text{also ist:}  ~~  z'' &= \left( \alpha \cdot z + \beta \cdot 1 \right) \cdot \frac{1}{z}  \\
	                          &= \alpha + \frac{\beta}{z}
\end{align*}
Wir gehen nun wie vorhin den Weg über ein LGS. Wir möchten $n$ auf $-1$
abbilden und $f$ auf $1$. Also:
\begin{align*}
	-1 &= \alpha + \frac{\beta}{n} ~ , \numberthis\label{zlgs1}  \\
	 1 &= \alpha + \frac{\beta}{f}  \numberthis\label{zlgs2}
\end{align*}
Aus \eqref{zlgs2} ergibt sich direkt:
\begin{align*}
	\alpha = 1 - \frac{\beta}{f}  \numberthis\label{zlgs3}
\end{align*}
Setze nun \eqref{zlgs3} in \eqref{zlgs1} ein:
\begin{align*}
	-1 &= 1 - \frac{\beta}{f} + \frac{\beta}{n}  \\
	   &= 1 - \frac{n \cdot \beta}{n \cdot f} + \frac{f \cdot \beta}{f \cdot n}  \\
	   &= 1 + \frac{\beta \cdot (f - n)}{n \cdot f}  \\
	\iff ~~ -2 &= \beta \cdot \frac{f - n}{n \cdot f}  \\
	\iff ~~ \beta &= \frac{-2 \cdot n \cdot f}{f - n}  \numberthis\label{zlgs4}
\end{align*}
Setze letztendlich \eqref{zlgs4} in \eqref{zlgs3} ein:
\begin{align*}
	\alpha &= 1 - \frac{2 \cdot n \cdot f}{f - n} \cdot \frac{1}{f}  \\
	       &= 1 + \frac{2 \cdot n}{f - n}  \\
	       &= \frac{f - n}{f - n} + \frac{2 \cdot n}{f - n}  \\
	       &= \frac{f - n + 2 \cdot n}{f - n}  \\
	       &= \frac{f + n}{f - n}
\end{align*}
Damit haben wir final $M_3$ bestimmt:
\begin{align*}
	M_3 &=
	\begin{pmatrix}
		\frac{-2n}{l - r} & 0 & \frac{l + r}{l - r} & 0  \\
		0 & \frac{-2n}{b - t} & \frac{b + t}{b - t} & 0  \\
		0 & 0 & \frac{f + n}{f - n} & \frac{-2nf}{f - n}  \\
		0 & 0 & 1 & 0
	\end{pmatrix}
\end{align*}

\section{Vereinfachung: Field of View $\vartheta$ und Aspect Ratio $a$}
Sehr häufig genügt es, den Bildbereich in der Projektionsebene zu
zentrieren. Damit blickt die Kamera genau entlang der negativen
$z$-Achse, ebenso herrscht Symmetrie und es ist immer $b = -t$ und $l =
-r$. Man muss die Kamera dann auch nicht mehr über $l$, $r$, $b$ und $t$
parametrisieren, sondern kann einfach einen Öffnungswinkel $\vartheta$
angeben. Möchte man keinen quadratischen Bildbereich, dann kann man das
Seitenverhältnis $a$ zusätzlich angeben.

$n$ und $f$ werden weiterhin angegeben.

\subsection{Field of View $\vartheta$}
Wir projizieren initial wie gehabt $y' = y \cdot \frac{n}{z}$. Da die
Division zu einem anderen "`Zeitpunkt"' passiert als die Multiplikation
mit $n$ (nämlich erst bei der Dehomogenisierung und nicht direkt als
Koeffizient für $y$), schreiben wir explizit:
\begin{align*}
	y' &= \left( y \cdot n \right) \cdot \frac{1}{z}
\end{align*}
Dann betrachten wir Abbildung~\ref{fovy}: $t$ ist nicht gegeben, wir
können es aber über den Öffnungswinkel bestimmen. Relevant ist dazu nur
das obere Dreieck mit dem eingezeichneten rechten Winkel.

\begin{figure}[h]
	\centering
	\begin{tikzpicture}
	\draw[shift={(-1.5, -1.5)},<->]
		(0, 0.5) node[above] {$y$}
		-- (0, 0)
		-- (-0.5, 0) node[left] {$z$};

	\begin{scope}[scale=0.5]
		\draw (0, 0) -- (7.5, 0);
		\draw[thick] (0, 0) -- (7.5, 5);
		\draw[thick] (0, 0) -- (7.5, -5);

		\draw (6, 4) -- (6, -4);

		\draw[fill=red] (0, 0) node[left] {$(0, 0)$} circle (2pt);
		\draw[fill=red] (6, 0) node[below left] {$n$} circle (2pt);

		\draw[fill=red] (6, 4) node[above left] {$t$} circle (2pt);
		\draw[fill=red] (6, -4) node[below left] {$-t$} circle (2pt);

		\draw[thick] (0, 0) ++ (33:2) arc (33:-33:2);
		\draw (0, 2) node[above] {$\vartheta$} to[bend left] (1, 0.25);

		\draw (6, 0) ++ (90:0.5) arc (90:180:0.5);
		\draw[fill=black,shift={(-0.2, 0.2)}] (6, 0) circle (2pt);
	\end{scope}
\end{tikzpicture}

	\caption%
	{Kamera mit $n$ und Öffnungswinkel $\vartheta$. Die zweite Ebene bei
	$f$ befindet sich weiterhin rechts von $n$ und ist nicht
	eingezeichnet.}

	\label{fovy}
\end{figure}

In diesem Dreieck gilt:
\begin{align*}
	\tan \frac{\vartheta}{2} &= \frac{t}{n}  \\
	\Rightarrow ~~ t &= n \cdot \tan \frac{\vartheta}{2}
\end{align*}
Ähnlich wie bei der allgemeinen Projektion haben wir damit die
Forderung:
\begin{align*}
	y' \in [-n \cdot \tan \frac{\vartheta}{2}, n \cdot \tan \frac{\vartheta}{2}]  ~~  \rightarrow  ~~  y'' \in [-1, 1]
\end{align*}
Wir können jetzt ausnutzen, dass der Bildbereich symmetrisch ist.
Dadurch müssen wir lediglich durch die Intervallgrenzen dividieren.
Also:
\begin{align*}
	y'' &= y' \cdot \frac{1}{n \cdot \tan \frac{\vartheta}{2}}  \\
	    &= y \cdot n \cdot \frac{1}{n \cdot \tan \frac{\vartheta}{2}} \cdot \frac{1}{z}  \\
	    &= y \cdot \frac{1}{\tan \frac{\vartheta}{2}} \cdot \frac{1}{z}
\end{align*}
Man beachte, dass dies ein reiner Koeffizient für $y$ ist und nichts
addiert wird. Das wird die Matrix weiter vereinfachen.

\subsection{Aspect Ratio $a$}
Die $x$-Richtung lässt sich prinzipiell analog zur $y$-Richtung
bestimmen. Unter Umständen will man aber keinen quadratischen
Bildbereich darstellen. Es genügt hier, das Seitenverhältnis von Breite
zu Höhe zu betrachten:
\begin{align*}
	a = \frac{w}{h}
\end{align*}
Um zu einem rechteckigen Bildbereich zu gelangen, werden wir $y''$ nicht
weiter ändern, sondern nur $x''$. Das heißt, der Öffnungswinkel
$\vartheta$ steuert nur die Öffnung der Kamera in $y$-Richtung,
wohingegen der Winkel in $x$-Richtung größer oder kleiner sein kann.

Wie verwerten wir nun $a$? Ganz anschaulich: Wir befinden uns mit einem
positiven $x'' = k \cdot r$ irgendwo auf der Strecke von $0$ bis $r$
oder darüberhinaus.\footnote{$r$ sei wie $t$ bestimmt und hat auch
denselben Wert.} Für $y$ ist klar, dass bei $k = 1$ das Ende des
Bildbereichs erreicht ist. Ist der Bildbereich nun aber beispielsweise
doppelt so breit wie hoch, dann kann $x''$ auch doppelt so weit laufen.
Um es auch in den Bereich $[-1, 1]$ zu skalieren, muss demnach durch $a$
dividiert werden.

\subsection{Endergebnis der vereinfachten Projektionsmatrix}
$x$ und $y$ haben wir wie eben dargestellt abgeändert, $z$ bleibt
gleich, ebenso wie die letzte Zeile der Matrix. Damit ergibt sich
insgesamt:
\begin{align*}
	M_4 &=
	\begin{pmatrix}
		\frac{1}{\tan \frac{\vartheta}{2}} \frac{1}{a} & 0 & 0 & 0  \\
		0 & \frac{1}{\tan \frac{\vartheta}{2}} & 0 & 0  \\
		0 & 0 & \frac{f + n}{f - n} & \frac{-2nf}{f - n}  \\
		0 & 0 & 1 & 0
	\end{pmatrix}
\end{align*}

\section{Hintergrund: Warum spielt $z''$ nach der Projektion überhaupt
eine Rolle?}
Hätte man einzig und alleine eine Matrix für die Kamera, dann wäre $z''$
nicht weiter wichtig. Man könnte damit sogar immer noch einen Depth
Buffer umsetzen, indem man sich die $z$-Koordinaten der projizierten
Punkte ansieht. In der Regel gibt es jedoch neben der Kameramatrix noch
viele weitere Transformationsmatrizen. Diese Matrizen werden in der
Regel auch nicht einzeln gespeichert, sondern miteinander multipliziert,
um eine gemeinsame Matrix zu erhalten, mit der alle Punkte multipliziert
werden können -- für die Umrechnung eines Punktes von Weltkoordinaten in
Bildkoordinaten bedarf es dann nur einer einzigen
Matrix-Vektor-Multiplikation. In dieser Matrix ist auch die
Kameramatrix impliziert enthalten. Um jetzt noch einen Depth Buffer
umsetzen zu können, muss das final projizierte $z''$ noch Aussagekraft
besitzen können: Wo in $z$-Richtung lag der projizierte Punkt
ursprünglich?

\end{document}
